%\documentclass[runningheads]{llncs}
\documentclass[]{llncs}
\usepackage{makeidx}

\begin{document}
\addtocmark{Southern Methodist University} % additional mark in the TOC

\title{LABORATORY EARTHQUAKE ANALYSIS}
%\subtitle{Optional Subtitle Goes Here}

\author{Olga Tanyuk\inst{1}, Daniel Davieau\inst{1}, Dr. Michael L. Blanpied\inst{1}, Dr. Charles South\inst{1} \and Dr. Daniel W. Engels\inst1}

\institute{Southern Methodist University, Dallas TX 75205, USA \and {\em Add Los Alamos, USGS and or Kaggle here?}}

\maketitle

\begin{abstract}
Earthquakes cause death and destruction.  The technologies used in the laboratory to simulate and collect earthquake data have improved over time. In this study we predict the time remaining until {\em laboratory} earthquakes occur more accurately(hopefully) than a 2017 Los Alamos National Laboratory study\cite{Bertrand}.  We analyze data patterns based on geophysical subject matter expertise, statistical methods and the latest technology available. We design a statistical algorithm to model the patterns and predict the time remaining until a laboratory earthquake will occur for given test data. We compare predicted versus actual time remaining to determine our accuracy.
We predicted impending laboratory earthquakes with {\em TBD accuracy, null hypothesis, statistical results with pvalue or confidence interval and relevent scores}.
The evidence of this experiment suggests {\em depends on final results}

\end{abstract}
\section{INTRODUCTION}
%1 Paragraph Motivtion (Sets General problem domain)
In August 2017 LANL conducted an experiment\cite{Bertrand} which predicted the remaining time until \emph{laboratory} earthquakes occur with 90\% accuracy. 

Subsequently there have been improvements in the technology used to collect and measure laboratory seismic signal data{\em (additional facts to be added?)}. There have also been improvements in computing power such as the software and hardware required for GPU computing. LANL is now hosting a Kaggle competition providing data {\em (presumably collected using improved technologies)} to the public.
\begin{quote}"If this challenge is solved and the physics are ultimately shown to scale from the laboratory to the field, researchers will have the potential to improve earthquake hazard assessments that could save lives and billions of dollars in infrastructure"\cite{kaggle}\end{quote}

%1 Paragraph Problem Statement (Specific Problem solved by the work)
Given seismic signal data with considerably more a-periodic laboratory earthquake failures and modern computing hardware; we improve on the Los Alamos study\cite{Bertrand} to determine when laboratory earthquakes will occur.

%2-3 paragraphs on solution \par

%1 Paragraph on main results (plural) \par

%1 Paragraph on main conclusions (plural) \par

%1 Paragraph on paper organization \par

\section{TUTORIAL MATERIAL}
Paper should be tutorial in nature
Audience is data scientists of varying levels of knowledge. Keep newer students in mind
\section{DATA}
Must have section that defines data
Use tables and figures to illustrate data attributes
\section{METHODS AND EXPERIMENTS}
Define algorithms, methods and eperiments
DO NOT give play by play of everything we did
Dont put code in paper; if anything put in appendix.
Put versions of software but nop one cares about how to use technology; just state what we did.
\section{RESULTS}
Results of experiments
Use tables and graphs
Use tables and graphs
Use tables and graphs
Don't forget explanations
\section{ANALYSIS}
Analyze results.
These are NOT conclusions.
\section{ETHICS}
Discuss ethics of your problem
You MUST have ethics section.
\section{CONCLUSION}
Draw conclusionS (plural, more than one conclusion- minimum of 3)
This is NOT a summary section.
\bibliographystyle{splncs}
\bibliography{OlgaDan}
\end{document}
