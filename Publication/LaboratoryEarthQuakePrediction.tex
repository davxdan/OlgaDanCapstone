%\documentclass[runningheads]{llncs}
\documentclass[]{llncs}
\usepackage{makeidx}

\begin{document}
\addtocmark{Southern Methodist University} % additional mark in the TOC

\title{LABORATORY EARTHQUAKE ANALYSIS}
%\subtitle{Optional Subtitle Goes Here}

\author{Olga Tanyuk\inst{1}, Daniel Davieau\inst{1}, Dr. Michael L. Blanpied\inst{1}, Dr. Charles South\inst{1} \and Dr. Daniel W. Engels\inst1}


\institute{Southern Methodist University, Dallas TX 75205, USA \and {\bf Add Los Alamos, USGS and or Kaggle here?}}

\maketitle

\begin{abstract}
In August 2017 the Los Alamos National Laboratory predicted\cite{Bertrand} the time remaining of imminent laboratory earthquakes to occur with 90 \% confidence and $ R^{2}=.89 $  {\em additional statistical facts to be added}\par

The technologies for collecting data and processing it have since improved

Earthquakes cause deaths and damage. \par
; predict the time remaining for imminent laboratory earthquakes more precisely than

We analyze the data for patterns using geological subject matter expertise, statistical methods and natural intuition. We design a statistical algorithm to model the patterns and predict the time remaining until a laboratory earthquake will occur for given test data. We compare predicted versus actual time remaining to determine our accuracy. \par
The evidence of this experiment suggests  {\em null hypothesis, statistical results with pvalue or confidence interval and releven t scores} we can predict impending laboratory earthquakes
{\em "Be careful not to accidentally plagurize. DO NOT use figures from other publications. Even if you cite it; you are getting into areas where copyright issues arise."}
\end{abstract}
\section{INTRODUCTION}
1 Paragraph Motivtion (Sets Genreral problem domain) \par
1 Paragraph P{roblem Statement (Specific Problem solved by the work) \par
2-3 paragraphs on solution \par
1 Paragraph on main results (plural) \par
1 Paragraph on main conclusions (plural) \par
1 Paragraph on paper organization \par
Data was attained from a Kaggle Competition sponsored by the Los Alamos National Laboratory:  www.kaggle.com/c/LANL-Earthquake-Prediction. The data in this competition is the result of a laboratory simulation.
\subsubsection*{This is another section.}
We assume that $H$ is $\left(A_{\infty}, B_{\infty}\right)$-subqua\-dra\-tic at infinity, for some constant \dots
\paragraph{Notes and Comments.} The first results on subharmonics were \dots
\section{TUTORIAL MATERIAL}
Paper should be tutorial in nature
Audience is data scientists of varying levels of knowledge. Keep newer students in mind
\section{DATA}
Must have section that defines data
Use tables and figures to illustrate data attributes
\section{METHODS AND EXPERIMENTS}
Define algorithms, methods and eperiments
DO NOT give play by play of everything we did
Dont put code in paper; if anything put in appendix.
Put versions of software but nop one cares about how to use technology; just state what we did.
\section{RESULTS}
Results of experiments
Use tables and graphs
Use tables and graphs
Use tables and graphs
Don't forget explanations
\section{ANALYSIS}
Analyze results.
These are NOT conclusions.
\section{ETHICS}
Discuss ethics of your problem
You MUST have ethics section.
\section{CONCLUSION}
Draw conclusionS (plural, more than one conclusion- minimum of 3)
This is NOT a summary section.
\bibliographystyle{splncs}
\bibliography{OlgaDan}
\end{document}
